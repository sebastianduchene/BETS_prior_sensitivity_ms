% Template for PLoS
% Version 3.6 Aug 2022
%
% % % % % % % % % % % % % % % % % % % % % %
%
% -- IMPORTANT NOTE
%
% This template contains comments intended 
% to minimize problems and delays during our production 
% process. Please follow the template instructions
% whenever possible.
%
% % % % % % % % % % % % % % % % % % % % % % % 
%
% Once your paper is accepted for publication, 
% PLEASE REMOVE ALL TRACKED CHANGES in this file 
% and leave only the final text of your manuscript. 
% PLOS recommends the use of latexdiff to track changes during review, as this will help to maintain a clean tex file.
% Visit https://www.ctan.org/pkg/latexdiff?lang=en for info or contact us at latex@plos.org.
%
%
% There are no restrictions on package use within the LaTeX files except that no packages listed in the template may be deleted.
%
% Please do not include colors or graphics in the text.
%
% The manuscript LaTeX source should be contained within a single file (do not use \input, \externaldocument, or similar commands).
%
% % % % % % % % % % % % % % % % % % % % % % %
%
% -- FIGURES AND TABLES
%
% Please include tables/figure captions directly after the paragraph where they are first cited in the text.
%
% DO NOT INCLUDE GRAPHICS IN YOUR MANUSCRIPT
% - Figures should be uploaded separately from your manuscript file. 
% - Figures generated using LaTeX should be extracted and removed from the PDF before submission. 
% - Figures containing multiple panels/subfigures must be combined into one image file before submission.
% For figure citations, please use "Fig" instead of "Figure".
% See http://journals.plos.org/plosone/s/figures for PLOS figure guidelines.
%
% Tables should be cell-based and may not contain:
% - spacing/line breaks within cells to alter layout or alignment
% - do not nest tabular environments (no tabular environments within tabular environments)
% - no graphics or colored text (cell background color/shading OK)
% See http://journals.plos.org/plosone/s/tables for table guidelines.
%
% For tables that exceed the width of the text column, use the adjustwidth environment as illustrated in the example table in text below.
%
% % % % % % % % % % % % % % % % % % % % % % % %
%
% -- EQUATIONS, MATH SYMBOLS, SUBSCRIPTS, AND SUPERSCRIPTS
%
% IMPORTANT
% Below are a few tips to help format your equations and other special characters according to our specifications. For more tips to help reduce the possibility of formatting errors during conversion, please see our LaTeX guidelines at http://journals.plos.org/plosone/s/latex
%
% For inline equations, please be sure to include all portions of an equation in the math environment.  For example, x$^2$ is incorrect; this should be formatted as $x^2$ (or $\mathrm{x}^2$ if the romanized font is desired).
%
% Do not include text that is not math in the math environment. For example, CO2 should be written as CO\textsubscript{2} instead of CO$_2$.
%
% Please add line breaks to long display equations when possible in order to fit size of the column. 
%
% For inline equations, please do not include punctuation (commas, etc) within the math environment unless this is part of the equation.
%
% When adding superscript or subscripts outside of brackets/braces, please group using {}.  For example, change "[U(D,E,\gamma)]^2" to "{[U(D,E,\gamma)]}^2". 
%
% Do not use \cal for caligraphic font.  Instead, use \mathcal{}
%
% % % % % % % % % % % % % % % % % % % % % % % % 
%
% Please contact latex@plos.org with any questions.
%
% % % % % % % % % % % % % % % % % % % % % % % %

\documentclass[10pt,letterpaper]{article}
\usepackage[top=0.85in,left=2.75in,footskip=0.75in]{geometry}

% amsmath and amssymb packages, useful for mathematical formulas and symbols
\usepackage{amsmath,amssymb}

% Use adjustwidth environment to exceed column width (see example table in text)
\usepackage{changepage}

% textcomp package and marvosym package for additional characters
\usepackage{textcomp,marvosym}

% cite package, to clean up citations in the main text. Do not remove.
\usepackage{cite}

% Use nameref to cite supporting information files (see Supporting Information section for more info)
\usepackage{nameref,hyperref}
\usepackage[normalem]{ulem}
% line numbers
\usepackage[right]{lineno}

% ligatures disabled
\usepackage[nopatch=eqnum]{microtype}
\DisableLigatures[f]{encoding = *, family = * }

% color can be used to apply background shading to table cells only
\usepackage[table]{xcolor}

% array package and thick rules for tables
\usepackage{array}

% create "+" rule type for thick vertical lines
\newcolumntype{+}{!{\vrule width 2pt}}

% create \thickcline for thick horizontal lines of variable length
\newlength\savedwidth
\newcommand\thickcline[1]{%
  \noalign{\global\savedwidth\arrayrulewidth\global\arrayrulewidth 2pt}%
  \cline{#1}%
  \noalign{\vskip\arrayrulewidth}%
  \noalign{\global\arrayrulewidth\savedwidth}%
}

% \thickhline command for thick horizontal lines that span the table
\newcommand\thickhline{\noalign{\global\savedwidth\arrayrulewidth\global\arrayrulewidth 2pt}%
\hline
\noalign{\global\arrayrulewidth\savedwidth}}


% Remove comment for double spacing
%\usepackage{setspace} 
%\doublespacing

% Text layout
\raggedright
\setlength{\parindent}{0.5cm}
\textwidth 5.25in 
\textheight 8.75in

% Bold the 'Figure #' in the caption and separate it from the title/caption with a period
% Captions will be left justified
\usepackage[aboveskip=1pt,labelfont=bf,labelsep=period,justification=raggedright,singlelinecheck=off]{caption}

\renewcommand{\figurename}{Fig}

% Use the PLoS provided BiBTeX style
\bibliographystyle{plos2015}

% Remove brackets from numbering in List of References
\makeatletter
\renewcommand{\@biblabel}[1]{\quad#1.}
\makeatother



% Header and Footer with logo
\usepackage{lastpage,fancyhdr,graphicx}
\usepackage{epstopdf}
%\pagestyle{myheadings}
\pagestyle{fancy}
\fancyhf{}
%\setlength{\headheight}{27.023pt}
%\lhead{\includegraphics[width=2.0in]{PLOS-submission.eps}}
\rfoot{\thepage/\pageref{LastPage}}
\renewcommand{\headrulewidth}{0pt}
\renewcommand{\footrule}{\hrule height 2pt \vspace{2mm}}
\fancyheadoffset[L]{2.25in}
\fancyfootoffset[L]{2.25in}
\lfoot{\today}

%% Include all macros below

\newcommand{\lorem}{{\bf LOREM}}
\newcommand{\ipsum}{{\bf IPSUM}}

%% END MACROS SECTION


\begin{document}
\vspace*{0.2in}

% Title must be 250 characters or less.
\begin{flushleft}
{\Large
\textbf\newline{Assessing the effect of model specification and prior sensitivity on Bayesian tests of temporal signal} % Please use "sentence case" for title and headings (capitalize only the first word in a title (or heading), the first word in a subtitle (or subheading), and any proper nouns).
}
\newline
% Insert author names, affiliations and corresponding author email (do not include titles, positions, or degrees).
\\
John H Tay\textsuperscript{1},
Arthur Kocher\textsuperscript{2,3},
Sebastian Duchene\textsuperscript{1,4,*},
\\
\bigskip
\textbf{1} Peter Doherty Institute for Infection and Immunity, Department of Microbiology and Immunology, University of Melbourne, Melbourne, Australia
\\
\textbf{2} Transmission, Infection, Diversification and Evolution Group, Max Planck Institute of Geoanthropology, Jena, Germany.
\\
\textbf{3} Department of Archaeogenetics, Max Planck Institute for Evolutionary Anthropology, Leipzig, Germany.
\\
\textbf{4} Department of Computational Biology, Institut Pasteur, Paris, France
\\
\bigskip

% Insert additional author notes using the symbols described below. Insert symbol callouts after author names as necessary.
% 

% \textcurrency b Insert second current address 
% \textcurrency c Insert third current address

% Use the asterisk to denote corresponding authorship and provide email address in note below.
* sduchene@pasteur.fr

\end{flushleft}
% Please keep the abstract below 300 words
\section*{Abstract}
Our understanding of the evolution of many microbes has been revolutionized by the molecular clock, a statistical procedure used to infer evolutionary rates and timescales from analysis of biomolecular sequences. In all molecular clock models, evolutionary rates and times are jointly unidentifiable and `calibration' information must therefore be used. 

For microbes that display high evolutionary rates, sequences sampled at different time points can be employed for such calibration. Before attempting so, it is recommended to verify that the data carry sufficient information for molecular dating, a practice referred to as temporal signal assessment. Recently, a fully Bayesian Evaluation of Temporal Signal (BETS) approach was proposed to overcome known limitations of other commonly used techniques such as root-to-tip regression or date randomisation tests. However, BETS requires the specification of a full Bayesian phylogenetic model, posing challenges for untangling the impacts of model choices on temporal signal detection. In this study, we aimed to (i) explore the effect of molecular clock model and tree prior specifications on BETS results and (ii) provide guidelines for improving our confidence in molecular clock estimates. 

Using empirical microbial molecular sequence data sets as well as simulations, we show that the tree prior can have a substantial impact on the accuracy of temporal signal assessment. In particular, we found that highly informative priors that are inconsistent with the data can result in the false detection of temporal signal and that this problem is more pronounced when using a strict clock model. In consequence, we recommend (i) using prior sensitivity analyses and prior predictive simulations to determine whether the prior is reasonable and the robustness of the inferences, (ii) including additional information in the form of internal node constraints or informative molecular clock rate distributions, and (iii) ensuring the the molecular clock model captures rate variation among lineages.
\newline

\textbf{Keywords:} Molecular clock, temporal signal, Bayesian phylogenetics, microbial evolution. 



% Please keep the Author Summary between 150 and 200 words
% Use first person. PLOS ONE authors please skip this step. 
% Author Summary not valid for PLOS ONE submissions.   
\section*{Author summary}
Molecular sequence data have become essential to infer evolutionary relationships and timescales of organisms via molecular clock models. In particular, our knowledge of when historical and modern pathogens emerged and spread is largely grounded on molecular clock models. These inferences, however, assume that sequence sampling times must have captured a sufficient amount of evolutionary change, which is typically determined using tests of temporal signal, such as the Bayesian Evaluation of Temporal Signal, BETS. Although BETS is generally effective, here we show that it can incorrectly detect temporal signal if the chosen evolutionary model makes implausible statements about the evolutionary timescale, a situation that is difficult to diagnose, particularly with complex Bayesian models. We demonstrate that this problem is due to a statistical artefact, that we refer to as tree extension and that it can be minimised by conducting careful prior predictive simulations, and by eliciting biologically plausible priors in the model. Overall, our study provides guidelines for improving our statistical confidence in estimates of evolutionary timescales, with key applications for recently emerging pathogens and data sets involving ancient molecular data.

\linenumbers

% Use "Eq" instead of "Equation" for equation citations.
\section*{Introduction}
Molecular sequence data have been essential to unravel the evolutionary history of many organisms. The molecular clock is a statistical procedure employed  to estimate the date of phylogenetic divergence events based on the hypothesis that molecular evolution, in the form of substitutions, follows an identifiable statistical process. For example, under the earliest and simplest molecular clock model, known as the strict clock, substitutions are assumed to accumulate at a constant rate over time and across lineages \cite{zuckerkandl1965evolutionary}. At the other end of the spectrum, relaxed molecular clocks allow every lineage in a phylogenetic tree to display a different evolutionary rate (\cite{drummond2006relaxed} and reviewed in \cite{ho2014molecular}). 

All molecular clock models have a fundamental limitation, where evolutionary rates and times are jointly unidentifiable. That is, there exist an infinite number of combinations of evolutionary rates and times that are compatible with a given amount of evolutionary divergence \cite{yang2006bayesian,dos2013unbearable}. For this reason, external information allowing to constrain some of the model's parameter must be used, a process known as a molecular clock calibration. 
%Too much detail, delete for conciseness: Consider two sequences whose genetic divergence from their most recent common ancestor is 10 subs/site. In the absence of calibrating information, it is impossible to know \textit{how rapidly} they evolve and \textit{when} they diverged. The calibration can be a known evolutionary rate, such as 1 subs/site/year, or a divergence date, such as 1 year before present. The genetic distance can be divided by the evolutionary rate to infer the divergence time to infer a time to the most recent common ancestor of 10 years, or the genetic distance can be divided by the divergence date to infer the evolutionary rate, 10/subs/site/year in this case.
The finding that some organisms accumulate substitutions in a measurable timescale prompted the use of sequence sampling times for calibration, a practice known as "tip calibration" \cite{rodrigo1999coalescent, korber2000timing}. The latter is particularly useful for microbial organisms for which fossil information cannot be used, or is not available, to constrain internal node dates. 
%The rationale of this practice is that sequence data collected at different points in time should have accumulated a corresponding number of substitutions. With the example above, a sequence collected six months of the common ancestor would have accumulated an expected number of 5 subs/site (10 subs/site/year $\times$ 0.5 years = 5 subs/site), whereas one collected after 1 year would have accrued on average 10 subs/site (10 subs/site/year$\times$ 1 year = 10 subs/site). As a result, sequence sampling times act as a time-calibration that is intuitively informative about the evolutionary rate.

A fundamental requirement for using tip calibration is that the sequenced data were sampled from a measurably evolving population, i.e. that the the interval of time over which the samples were taken captures an appreciable amount of evolutionary change in the studied organism \cite{drummond2003measurably}. For rapidly evolving pathogens, such as RNA viruses, this might already be achieved by drawing samples over weeks or months. For more slowly evolving microbes sampling over many years or centuries may be needed. 

There exist several statistical tests to determine whether a sampled population is measurably evolving behaviour, also known as tests of temporal signal. The root-to-tip regression takes a phylogenetic tree for which the branch lengths measure evolutionary distance (i.e. a phylogram) and fits a linear regression of the distance from the root to the tips as a function of their sampling time \cite{korber2000timing}. The regression slope is a crude estimate of the evolutionary rate, the \textit{x-}intercept is the time to the most recent common ancestor, and the $R^2$ is a measure of clocklike evolution. In general, the root-to-tip regression is a powerful tool for visual inspection of the data, for example to detect outliers or identify lineages with particularly low or high evolutionary rates \cite{rambaut2016exploring,featherstone2023clockor2,volz2017scalable}. However, because the data points are not statistically independent, resulting statistics such as \textit{p-}values are invalid and cannot be used as formal statistical tests of temporal signal. \cite{rieux2016inferences}. A different approach, referred to as the `date randomisation test', consist of fitting a molecular clock to the data after permuting the sampling times multiple times to obtain a `null' distribution of the evolutionary rate \cite{ramsden2009hantavirus}. The data are considered to have temporal signal if the evolutionary rate estimated with the correct sampling times falls outside such `null' distribution \cite{ramsden2009hantavirus,duchene2015performance}.

In order to overcome the issues associated with the above-mentioned techniques, a fully Bayesian Evaluation of Temporal Signal (BETS) approach was recently proposed \cite{duchene2020bayesian}.  The premise of this test is that the data sampled from a measurably evolving population should have higher statistical fit when the sampling times are included than when they are not, which can be assessed through model selection. In practice, the data are analysed with their correct sampling times and with all samples assigned the same date (i.e. at present), while keeping the rest of the phylogenetic model the same, including the molecular clock, tree prior and substitution model. The log marginal likelihood is calculated in each case to compute log Bayes factors, which quantify the amount of evidence for one model over another, here that with sampling times vs that without. A major advantage of BETS is that it can consider the full model and it naturally accommodates important sources of uncertainty, including that due to radio carbon dating of ancient DNA studies \cite{molak2015empirical}. 

Most parameters of the phylogenetic model have individual prior probability distributions that can be chosen by the user, for example, the evolutionary rate, or the transition-to-transversion ratio of the HKY substitution model. The phylogenetic tree topology and branch lengths are usually assigned a branching model, such as a coalescent or birth-death process. These tree priors implicitly impose a prior probability distribution on the ages of nodes, and therefore may inadvertently impose highly informative calibration priors. Moreover, model selection, as used for BETS, can be sensitive to the choice of prior, even if the posterior is not \cite{gelman1995avoiding, gelman2014bayesian}. Here we investigate the impact of the tree prior and associated parameters, and the molecular clock model impact the detection of temporal signal under a range of conditions. We also describe alternative parameterisations of the full Bayesian model that can improve the accuracy of tests of temporal signal.

\section*{Results}
\subsection*{Empirical data analyses}
We first explored the effect of model specification on BETS results using empirical datasets for which temporal signal was previously detected using other methods. The following three data sets were used: \textit{Vibrio cholerae} \cite{devault2014second}, the bacterium responsible for cholera; \textit{Powassan virus (POWV)} \cite{majander2020ancient}, a tick-borne virus; and \textit{Treponema pallidum} \cite{vogels2023phylogeographic}, the bacterium that causes syphilis. The \textit{V. cholerae} and \textit{T. pallidum} data sets include ancient samples, and the phylogenetic trees from the POWV and \textit{T. pallidum} indicate complex population structure that is typical of data sets from multiple outbreaks. We analysed the data sets using BETS under a coalescent tree prior with constantpopulation size and two possible clock models; a strict and an uncorrelated relaxed clock with an underlying log-normal distribution. Our choice the constant-size coalescent tree prior is based on statistical convenience, as it is fully parametric, but it does not necessarily model the biological process. We set up our analyses in BEAST1.10 \cite{suchard2018bayesian} and calculated log marginal likelihoods with and without sampling times for each combination of molecular clock model and tree prior. 

To assess the impact of the tree prior we considered different (hyper) prior distributions for the effective population size, $\theta$, the only parameter in the constant-size coalescent. In the exponential-size coalescent, which we also considered in our simulations (see below), this parameter is known as the `scaled population size' (denoted with the Greek letter $\Phi$) and it is proportional to the population size at present \cite{boskova2014inference}. This parameter is referred to as a \textit{scale parameter} for time because large values imply more dispersion (the molecular clock rate is also a scale parameter), and it is typically assigned a $1/x$ prior distribution, which is the Jeffrey's prior that is uninformative and invariant to reparameterisation  \cite{drummond2002estimating}. This prior has attractive attributes because it maximises the signal from the data, but it is an improper distribution (it does not integrate to one over its domain, because $\int_{0}^{\infty} \frac{1}{x}$ is undefined), a problem for model comparison using Bayes factors, because marginal likelihood calculations require that all priors be proper distributions \cite{r2019marginal, baele2013proper}. Instead, we selected three prior distributions, an exponential, $\Gamma$, and log-normal, that have been used in recent literature as shown in Table \ref{table:prior_distros_on_Phi}.

Our rationale for using different prior distributions on $\theta$ is its impact on parameters that pertain to the molecular clock. In particular, under the coalescent process, the expected time of divergence between sampled lineages is inversely proportional to the population size, meaning that large values of $\theta$ will result in an older time to the most recent common ancestor than small values for this parameter.

The prior on $\theta$ will also have an impact on the evolutionary rate for two key reasons. First, by impacting the overall age of the tree, it impacts the length of time over which the sequence data evolved. Second, the default prior for the evolutionary rate in BEAST1.10 is a Gamma ($\Gamma$) distribution with shape ($\alpha$) of 0.5 and beta ($\beta$, also known as the `rate') equal to the tree length (sum of all branch lengths) \cite{wang2014priors, ferreira2008bayesian}. In this software, this prior is known as the CTMC-rate reference prior and its mean value is $0.5 /$tree length \cite{ferreira2008bayesian}, meaning that it is indirectly impacted by $\theta$ (or $\Phi$). 

\begin{table}[h]
\caption{Prior distributions for the effective population size of the constant-size coalescent (known as $\theta$ in the constant-size coalescent and different to the scale parameter of the $\Gamma$ distribution).}
\begin{center} 
	\label{table:prior_distros_on_Phi}
	\begin{tabular}{| c + c |}
    \hline
		\multicolumn{1}{|c|}{Probability distribution function} & Parameters\\ \thickhline
		Exponential & mean, $\mu=1.0$\\ \hline
		Log-normal & mean, $\mu=1.0$; standard deviation $\sigma=5.0$\\ \hline
		$\Gamma$ (Gamma) & shape, $\kappa=0.001$; scale, $\theta=1000$\\ \hline
	\end{tabular}
\end{center}
\end{table}

The \textit{V. cholerae} data set displayed overwhelming support for temporal signal (Table \ref{table:empirical_bayes_factors} and Fig \ref{figure:polygon_plots}), regardless of the molecular clock model and prior on $\theta$, with log Bayes factors of over 200. Note that a log Bayes factor of 3.2 corresponds to a model posterior probability $\approx$0.95 \cite{tay2023detecting}, and is considered as `very strong support', following Kass and Raftery \cite{kass1995bayes}. Although in this data set the prior on $\theta$ did not impact model selection for detecting temporal signal, it did impact the magnitude of the Bayes factors.

For our other two data sets the impact of the prior on model selection was evident. For \textit{Poawassan virus} the $\Gamma$ and log-normal priors on $\theta$ suggested strong temporal signal, whereas the exponential prior strongly favoured the exclusion of sampling times, according to the strict and relaxed molecular clock models. In our analyses of the \textit{T. pallidum}  data set we found support for temporal signal under the strict molecular clock, according to all priors on $\theta$, although with very strong evidence only for the exponential prior and `positive evidence' for the $\Gamma$ and log-normal priors. Under the relaxed molecular clock model all priors had very strong support against temporal signal.

\begin{table}[h]
    \caption{Log Bayes factors between isochronous and heterochronous models for each dataset, separated by prior on effective population size, $\theta$}
    \begin{center}
    \label{table:empirical_bayes_factors}
    \begin{tabular}{| c + c | c | c |}
    \hline
    \multicolumn{1}{|c|}{\bf Species; Clock Model} & Exponential & Gamma & Log-normal\\ \thickhline
    \textit{Vibrio cholerae}; $Strict$ $Clock$ & 355.18 & 379.63 & 382.10 \\ \hline
    \textit{Vibrio cholerae}; $Relaxed$ $Clock$ & 208.97 & 439.63 & 219.60 \\  \hline
    \textit{Powassan virus (POWV)}; $Strict$ $Clock$ & -80.63 & 32.67 & 50.29 \\ \hline
    \textit{Powassan virus (POWV)}; $Relaxed$ $Clock$ & -221.94 & 18.79  & 27.23 \\ \hline
    \textit{Treponema pallidum}; $Strict$ $Clock$ & 105.80 & 2.17 & 1.85 \\ \hline
    \textit{Treponema pallidum}; $Relaxed$ $Clock$ & -34.37 & -1474.14 & -34.04 \\ \hline
    \end{tabular}
    \end{center}
\end{table}

\begin{figure}
	\begin{center}
		\includegraphics[width=14cm]{sandbox_figures/polygon_plot.pdf}\newline
		\vspace{-0.5cm}
		\caption{\textbf{Relative log marginal likelihoods of empirical data sets}. The polygons represent the relative log marginal likelihoods of each microbe dataset under a different effective population size ($\theta$) prior, analysed with four different configurations. Het (heterochronous) includes sampling, while iso (isochronous) does not include any sampling times. SC is strict clock and UCLD is the uncorrelated lognormal relaxed clock. Red represents an exponential hyperprior on the effective population size, blue is a $\Gamma$ hyperprior, and green is a log-normal hyperprior.}
		\label{figure:polygon_plots}
	\end{center}
\end{figure}

Our empirical data analyses overall demonstrate that the choice of prior has an impact on Bayesian model selection and support. In the case of the \textit{V. cholerae} data, the detection of temporal signal was relatively robust to the prior. Consistently, we found that the posterior distribution of key parameters such as the evolutionary rate was not very sensitive to the prior for this data set (Supplementary material). Moreover, this data set has been shown to have clear clocklike behaviour in root-to-tip regressions and date randomisation tests \cite{duchene2016genome}. In contrast, for the data sets of \textit{Powassan virus} and \textit{T. pallidum}, where temporal signal was not supported for all model configurations, we found that the posterior was more sensitive to the choice of prior (Supporting information).

\subsection*{Simulation experiments}
To understand the impact of the prior on BETS results in more details, we conducted a set of simulation experiments where the data generating process is well understood. We conducted simulations under four possible conditions: a strict or relaxed molecular clock, and where the phylogenetic time-trees were heterochronous or isochronous. Data from heterochronous trees are sampled from a measurably evolving population and are expected to display temporal signal, whereas those from isochronous trees are not from a measurably evolving population (the time-trees are ultrametric) and should not display temporal signal. 

For data generated under heterochronous time-trees, we found that ten out of ten simulation replicates were correctly classified as having temporal signal, using a log Bayes factor of at least 3.2 (Table \ref{table:heterochronous_simulations_unbounded} and Fig \ref{figure:heterochronous_polygons}). This perfect classification, which can be considered a low type II error (where a type II error is failing to support temporal signal when it is truly present), was supported regardless of the prior on $\theta$ and the molecular clock model. 

\begin{table}[h!]
	\caption{\textbf{Correctly classified simulation replicates under heterochronous trees.} A total of ten simulations were generated in each case, under heterochronous trees, such that they are expected to display temporal signal. A number of ten represents perfect classification according to the Bayesian evaluation of temporal signal, BETS and a log Bayes factor of at least 3.2 (strong evidence for temporal signal). The rows correspond to three possible priors on the effective population size of the constant-size coalescent, $\theta$. The `Best clock model' is a situation where we consider the best heterochronous and isochronous model, take their log Bayes factor, and determine temporal signal if it is at least 3.2.}
	\begin{center}
		\label{table:heterochronous_simulations_unbounded}
		\begin{tabular}{| c + c | c | c |}
			\hline
			\multicolumn{1}{|c|}{\bf True clock model; clock model in analysis} & Exponential & $\Gamma$ & Log-normal\\ \thickhline
			Strict clock; Strict clock     & 10 & 10 & 10 \\ \hline
			Strict clock; Relaxed clock    & 10 & 10 & 10 \\ \hline
			Strict clock; Best clock model & 10 & 10 & 10 \\ \hline
			Relaxed clock; Strict clock    & 10 & 10 & 10 \\ \hline
			Relaxed clock; Relaxed clock    & 10 & 10 & 10 \\ \hline
			Relaxed clock; Best clock model & 10 & 10 & 10 \\ \hline		
		\end{tabular}
	\end{center}
\end{table}

\begin{figure}[!h]
	\begin{center}
		\includegraphics[width=14cm]{sandbox_figures/het_sims.pdf}\newline
		\vspace{-0.5cm}
		\caption{\textbf{Relative log marginal likelihoods of simulations with temporal signal.} The polygons represent the relative log marginal likelihood under three possible priors on the effective population size ($\theta$) parameter of the constant-size coalescent tree prior. Each corner corresponds to a combination of model and sampling times, either a strict (SC) or relaxed molecular clock with an underlying log-normal distribution (UCLD), and with (heterochronous) or without (isochronous) sampling times. The correct model used to generate the data is the SC heterochronous. Each polygon is for one simulation replicate (a total of ten) and the colours denote whether we employed a hard bound on the root height of the form Uniform(0.0, 5.0), in blue, or not, in orange.} 
		\label{figure:heterochronous_polygons}
	\end{center}
\end{figure}

For our data generated under isochronous trees, which by definition have no temporal signal, we found perfect classification under the exponential prior on $\theta$ under both clock models (Table \ref{table:isochronous_simulations_unbounded} and Fig \ref{figure:ultrametric_polygons}). For most analyses under the $\Gamma$ and log-normal priors on $\theta$ we found that BETS incorrectly supported the presence of temporal signal. The exceptions were for analyses of data analysed under a relaxed clock, whether they were simulated under a strict or relaxed clock. 

A perplexing result occurs under the best molecular clock model for the $\Gamma$ and log-normal priors on $\theta$. Here we take the log Bayes factor of the best heterochronous model vs the best isochronous model, which produced an increase in the classification error, relative to using the relaxed clock only. This phenomenon occurs because the incorrect inclusion of sampling times can mislead molecular clock model selection. As a case in point one of the simulation replicates under a relaxed molecular clock and an isochronous tree (with no temporal signal) had the following log marginal likelihoods; -4109.87 for the heterochronous analyses with a strict clock, -4117.06 for the isochronous analyses with a strict clock, -4124.35 for the heterochronous analyses with a relaxed clock, and -4118.29 for the isochronous analyses with a relaxed clock. The log Bayes factors under the relaxed clock have strong evidence against temporal signal (log Bayes factor=-6.06 for heterochronous vs isochronous), whereas the opposite is true for the strict clock (log Bayes factor=7.19). However, the best heterochronous model has substantially stronger support than the best isochronous model (here the strict or relaxed molecular clock, whose log marginal likelihoods differ by only 1.3 units). It is also worthwhile to note that in general, analyses under the relaxed clock tended to have fewer classification errors (for data sets with no temporal signal) than the strict clock, regardless of the true molecular clock model used to generate the data.

\begin{table}[h!]
	\caption{\textbf{Correctly classified simulation replicates under isochronous trees.} A total of ten simulations were generated in each case, under isochronous trees, such that they are not expected to support temporal signal. A number of ten represents perfect classification according to the Bayesian evaluation of temporal signal, BETS and a log Bayes factor of at most -3.2 (strong evidence against temporal signal). The rows correspond to three possible priors on the effective population size of the constant-size coalescent, $\theta$. The `Best clock model' is a situation where we consider the best heterochronous and isochronous model, take their log Bayes factor, and determine a lack of temporal signal if it is at most -3.2.}
	\begin{center}
		\label{table:isochronous_simulations_unbounded}
		\begin{tabular}{| c + c | c | c |}
			\hline
			\multicolumn{1}{|c|}{\bf True clock model; clock model in analysis} & Exponential & $\Gamma$ & Log-normal\\ \thickhline
			Strict clock; Strict clock     & 10 & 0 & 0 \\ \hline
			Strict clock; Relaxed clock    & 10 & 10 & 10 \\ \hline
			Strict clock; Best clock model & 10 & 0 & 0 \\ \hline
			Relaxed clock; Strict clock    & 10 & 0 & 0 \\ \hline
			Relaxed clock; Relaxed clock    & 10 & 9 & 9 \\ \hline
			Relaxed clock; Best clock model& 10 & 0 & 1 \\ \hline		
		\end{tabular}
	\end{center}
\end{table}

\begin{figure}[!h]
	\begin{center}
		\includegraphics[width=14cm]{sandbox_figures/iso_sims.pdf}\newline
		\vspace{-0.5cm}
		\caption{\textbf{Relative log marginal likelihoods of simulations with no temporal signal.} The polygons here represent the same information as in Fig \ref{figure:heterochronous_polygons}. The correct model used to generate the data here is the SC isochronous (ultrametric).} 
		\label{figure:ultrametric_polygons}
	\end{center}
\end{figure}

Our simulation results demonstrate that detecting temporal signal when it is not present, which is akin to a type I error, is more common under some prior configurations (here the $\Gamma$ and log-normal priors on $\theta$) than the opposite (failing to detect temporal signal when it is present). Upon inspecting the resulting phylogenetic trees and the posterior of key parameters we found a probable cause. The incorrect inclusion of sampling times produces a dramatic overestimation of the height of the tree, especially under the strict molecular clock model, a phenomenon that we refer to as `tree extension' (Fig \ref{figure:ultrametric_tree_distortion}). Under this situation, the sampling times represent such a small proportion of the root height (the time from the most recently sampled node and the root-node), that the heterochronous tree is indistinguishable from one that is ultrametric and thus their log marginal likelihoods of a model with sampling times can be comparable, or higher than that without sampling times. In fact, the sampling times here spanned at most 0.5 units of time, such that they represent only 0.05\% of the total height of a tree height of 1,000, which is not unusual when there is tree extension. The phenomenon of tree extension also occurs under the relaxed molecular clock, but to a much lesser extent, and thus under this model it is easier to correctly classify isochronous data sets. 

\begin{figure}[h!]
	\begin{center}
		\includegraphics[width=15cm]{sandbox_figures/tree_distortion_ultrametric.pdf}\newline
		\vspace{-0.5cm}
		\caption{\textbf{Phylogenetic tree extension for a simulation replicate with no temporal signal}. Highest clade credibility trees from a data set simulated with no sampling times (isochronous) and under a strict molecular clock model (SC). The prior on $\theta$ is a $\Gamma(\kappa=0.001, \theta=1000)$, which resulted in high classification errors using BETS. The y-axis is the time from the present. Tip nodes have solid grey circles. Including sampling times that span 0.5 units of time and about 1/4 of the true root height induces dramatic overestimation of the root height, compared to the true model (SC, isochronous). This effect  occurs under both molecular clock models, the SC and the relaxed molecular clock with an underlying log-normal distribution (UCLD), but it is markedly less pronounced in the UCLD. Note that the y-axis is in logarithmic scale (log$_{10}$).}
		\label{figure:ultrametric_tree_distortion}
	\end{center}
\end{figure}


A problem with fixing the clock rate to 1.0 for the isochronous is that the branch lengths of the tree will be in units of subs/site. The $\theta$ parameter of the constant-size coalescent is proportional to units of time \cite{ho2011skyline, drummond2002estimating} (as are most other parameters of the tree prior, for example the growth rate of the exponential-growth coalescent). When time-calibrations are used $\theta$ is the population size multiplied by the generation time ($N_e \times$ generation time). Thus, the prior on $\theta$ for the heterochronous and isochronous analyses has different meanings (the branch lengths are in different units), unless this parameter is scaled to match the units of the branch lengths, or the molecular clock rate in the isochronous analyses is fixed to a plausible value. In our simulations we fixed the molecular clock rate to its true value, but we also found that using a number within the expected order of magnitude of the organism in question is sufficient (e.g. 10$^{-4}$ to 10$^{-3}$ subs/site/year for a ssRNA virus).

\begin{figure}[!h]
	\begin{center}
		\includegraphics[width=15cm]{sandbox_figures/tree_distortion_heterochronous.pdf}\newline
		\vspace{-0.5cm}
		\caption{\textbf{Phylogenetic trees from a simulation replicate with temporal signal}. Highest clade credibility trees from a data set simulated with sampling times (heterochronous) and under a strict molecular clock model (SC). The isochronous trees are inferred by fixing the molecular clock rate to the true value, such that the timescale is in the comparable units to the heterochronous analyses. Unlike the estimates for isochronous trees (e.g. Fig \ref{figure:ultrametric_tree_distortion}), the height of the trees  under all scenarios are comparable. Axes and labels are the same as those of Fig \ref{figure:ultrametric_tree_distortion}}
		\label{figure:heterochronous_tree_distortion}
	\end{center}
\end{figure}

Our results indicate that using priors favouring plausible node heights is important to ensure the accuracy of temporal signal detection. However, the interplay between the parameters of the tree prior and the resulting tree topologies and node heights is not necessarily trivial, particularly when the tree prior involves multiple parameters. Thus, defining suitable hyperpriors for the parameters of the tree prior requires careful attention. A pragmatic solution is to include additional prior information in the form of hard bounds on the root-height or the molecular clock rate. To this end, we investigated the effect of including a uniform prior between 0.0 and 5.0 units of time for the root height, meaning that trees that are older than 5.0 units have a prior probability of 0.0. Importantly, the trees under which we generated our simulations had root heights of around 2.0 units of time, and thus the hard bound of 5.0 allows for trees that are over twice as old as the truth. 

Setting hard bounds on the root height resulted in perfect classification accuracy for both, the heterochronous and isochronous simulations (Table \ref{table:heterochronous_simulations_bounded} and Fig \ref{figure:heterochronous_polygons}, and Table \ref{table:isorochronous_simulations_bounded} and Fig \ref{figure:ultrametric_polygons}, respectively). The improvement in classification for data sets with no temporal signal is likely because the hard bounds prevent the tree extension phenomenon, and thereby including sampling times imposes a penalty on the log marginal likelihood (see polygons in Fig \ref{figure:heterochronous_polygons}). Our empirical data set of \textit{V. cholerae}, which had evidence of temporal signal under all prior conditions, also displayed strong evidence for temporal signal with a hard bound on of 500 years before present on the root height (all log Bayes factors were at least 200 in favour of temporal signal).

\begin{table}[h!]
	\caption{\textbf{Correctly classified simulation replicates under heterochronous trees using hard bounds on the root height.} Rows and columns are identical to those of Table \ref{table:heterochronous_simulations_unbounded}, but here the heterochronous analyses include an explicit prior on the root height, via a uniform distribution between 0 and 5.0.}
	\begin{center}
		\label{table:heterochronous_simulations_bounded}
		\begin{tabular}{| c + c | c | c |}
			\hline
			\multicolumn{1}{|c|}{\bf True clock model; clock model in analysis} & Exponential & $\Gamma$ & Log-normal\\ \thickhline
			Strict clock; Strict clock     & 10 & 10 & 10 \\ \hline
			Strict clock; Relaxed clock    & 10 & 10 & 10 \\ \hline
			Strict clock; Best clock model & 10 & 10 & 10 \\ \hline
			Relaxed clock; Strict clock    & 10 & 10 & 10 \\ \hline
			Relaxed clock; Relaxed clock    & 10 & 10 & 10 \\ \hline
			Relaxed clock; Best clock model & 10 & 10 & 10 \\ \hline		
		\end{tabular}
	\end{center}
\end{table}

\begin{table}[h!]
	\caption{\textbf{Correctly classified simulation replicates under isochronous trees using hard bounds on the root height.} Rows and columns are identical to those of Table \ref{table:isochronous_simulations_unbounded}, but here the heterochronous analyses include an explicit prior on the root height, via a uniform distribution between 0 and 5.0.}
	\begin{center}
		\label{table:isorochronous_simulations_bounded}
		\begin{tabular}{| c + c | c | c |}
			\hline
			\multicolumn{1}{|c|}{\bf True clock model; clock model in analysis} & Exponential & $\Gamma$ & Log-normal\\ \thickhline
			Strict clock; Strict clock     & 10 & 10 & 10 \\ \hline
			Strict clock; Relaxed clock    & 10 & 10 & 10 \\ \hline
			Strict clock; Best clock model & 10 & 10 & 10 \\ \hline
			Relaxed clock; Strict clock    & 10 & 10 & 10 \\ \hline
			Relaxed clock; Relaxed clock    & 10 & 10 & 10 \\ \hline
			Relaxed clock; Best clock model& 10 & 10 & 10 \\ \hline		
		\end{tabular}
	\end{center}
\end{table}

The constant-size coalescent tree prior is statistically convenient, but likely unrealistic for many data sets. The exponential-size coalescent is an alternative in which population size changes deterministically (for its description in a phylodynamic context see \cite{volz2012complex}). This model has two key parameters, the `scaled population size' ($\Phi$) and the exponential `growth rate' ($r$). Due to its coalescent nature the time to the most recent common ancestor scales positively with $\Phi$. We investigated the performance of BETS under this tree prior, using the same three prior distributions on $\Phi$ as we did for $\theta$ in the constant-size coalescent and for the growth rate we used $Laplace(\mu=0.0, b=1.0)$. The relative log marginal likelihoods between models were very similar to those using the constant-size coalescent. Heterochronous data sets were overwhelmingly classified as having temporal signal, whereas in isochronous data sets using hard bounds correctly improved support for a lack of temporal signal (Supporting information). 

\subsection*{Prior predictive simulations and parameter correlations}
Our finding that the tree prior can impact model selection prompted an investigation of parameter interactions and of the expectations under the prior. We simulated phylogenetic trees from a prior distribution to inspect the correlation of parameters and the marginal prior for those that have obvious associations (e.g. the evolutionary rate has a CTMC-rate reference prior that depends on tree length). These simulations are commonly known as prior predictive simulations (e.g. \cite{wesner2021choosing}), and referred to in the phylogenetic literature as `sampling from the prior' \cite{nascimento2017biologist}. Initially, we set a uniform prior on $\theta$ from 0 to 10$^3$ and recorded the evolutionary rate, the tree length and root height. This prior is not generally recommended \cite{bouckaertDating}, and we present it here to illustrate correlations between parameters and the marginal prior, instead of using it for analysing our data.

An obvious finding is a natural positive correlation between tree length, root height, although the trends is not strictly linear (Fig \ref{figure:correlation_plots}). We also observed a positive correlation between $\theta$, tree length and tree height, which is expected because large population sizes impose long coalescent times \cite{rosenberg2002genealogical}. The nature of this correlation is heteroskedastic, with the variance in these tree statistics increasing with $\theta$. Our simulations demonstrate an inverse relationship between the evolutionary rate, $\theta$ and the tree statistics, but with a range that can span several orders of magnitude and with values ranging from 10$^{-9}$ to 10$^{-4}$ subs/site/time, meaning that the CTMC-rate reference prior tends to be diffuse. It is also noteworthy that the uniform prior on $\theta$ does not result in uniform marginal prior distributions for any of the parameters investigated here. These results illustrate the importance of visualising the prior and the fact that it is difficult to predict how different parameters will interact.

\begin{figure}[!h]
		\begin{center}
		\includegraphics[width=14.7cm]{sandbox_figures/prior_predictive_plots.pdf}\newline
		\vspace{-0.5cm}
  \newline
		\caption{\textbf{Marginal prior distributions and pairs plots.} The grey histograms for correspond to the parameter labelled at the bottom of each column; effective population size (pop. size, $\theta$), tree length, root height, and the evolutionary rate (evol. rate). The prior for $\theta$ here is a Uniform distribution between 0 and 10$^3$, while that for the evolutionary rate is a CTMC-rate reference prior. Note that the tree length and root height have units of time, the evolutionary rate is in subs/site/time, and $\theta$ is proportional to units of time. }
        \label{figure:correlation_plots}
		\end{center}
\end{figure}

We conducted prior predictive simulations for the six prior configurations for $\theta$ under a heterochronous data set. In Fig \ref{figure:prior_tree_distros} we show the resulting distribution on the root height. The exponential ($\theta \sim Exponential(\mu=1.0)$, $\Gamma$ ($\Gamma(\kappa=0.001, \theta=1000)$) and log-normal (log-normal$(\mu=1.0, \sigma=5.0)$) prior distributions respectively produced mean root heights of 2.90 (95\% quantile range, qr: 1.62 to 14.34), 1.61 (95\% qr: 1.50 to 4.87) and 772.11 (95\% qr: 1.66 to $5.06 \times 10^5$) units of time. Clearly, the log-normal is the most vague prior here, but it produces implausible values of several orders of magnitude higher than our expectation of tree heights of around one and ten units of time (as simulated using $\theta=1.0$).

For comparison, we also simulated trees under the same priors on $\theta$, with hard bounds on the root height, and with a uniform distribution with minimum and maximum values of 0.0 and 5.0, as described above. In this case, the exponential prior on $\theta$ yielded trees with root heights of mean 2.47 (95\% qr: 1.62 to 5.16), whereas those for the log-normal and $\Gamma$ were 1.61 (95\% qr: 1.50 to 5.11), and 1.96 (95\% qr: 1.54 to 4.42), respectively. As a result, in the log-normal prior on $\theta$ the use of hard bounds on the root height resulted in much smaller root heights, with a smaller impact on the exponential and $\Gamma$ priors.

\begin{figure}[!h]
	\begin{center}
		\includegraphics[width=13cm]{sandbox_figures/prior_tree_distros.pdf}\newline
		\vspace{-0.5cm}
		\caption{\textbf{Prior predictive simulations and marginal priors of root heights, given the prior on the effective population size, $\theta$}. Each panel corresponds to a different prior on $\theta$, as described on Table \ref{table:prior_distros_on_Phi} (Exponential($\mu=1.0$), Log-normal($\mu=$1.0, $\sigma=$5.0), and $\Gamma(\kappa=0.001, \theta=1000)$). We show five simulated trees from our analysis using sampling times (heterochronous analyses) and overlaid them (similar to densitree plots \cite{bouckaert2010densitree}). The violin plots show the prior densities of the root height and the hollow circles denote 100 randomly drawn samples from the prior. The y-axis is the time from the present, but note that the scales are different. Tip nodes are shown with solid grey circles. The densities and trees on the left, in orange, do not include an explicit prior on the root height ($T_h$), while those to the right, in purple, have a hard bound on tree height in the form of a uniform prior between 0.0 and 5.0 units of time.}
		\label{figure:prior_tree_distros}
	\end{center}
\end{figure}

\section*{Discussion}
Our study demonstrates that the choice of prior distribution has an substantial impact on Bayesian model selection and that it can mislead tests of temporal signal. In general, data sets with no temporal signal are easily misclassified when the prior favours an implausibly old root height and low evolutionary rates, resulting in type I errors. In turn, incorrectly detecting temporal signal appears to result in a systematic overestimation of the evolutionary timescale and an underestimation of the molecular clock rates.

We find that tree extension is the most probable reason for the incorrect detection of temporal signal in BETS, because it reduces the sampling window relative to the root height, such that the trees with sampling times are very similar to ultrametric trees. The phylogenetic likelihood of such overly old trees can be consistent with that from much shallower trees with no sampling times if their evolutionary rate is very low, such that the genetic distance along their branches is comparable. 

Our observation of tree extension pertains to isochronous data sets analysed with sampling times, but a similar situation occurs in date-randomisation tests \cite{ramsden2009hantavirus}. Here the sequence sampling times are permuted a number of times and the evolutionary rate is estimated. The data are considered to have temporal signal if the estimates from the permutations do not overlap with that from the correct sampling times. Notably, when the data do have temporal signal, the estimates from the permutations are substantially lower, implying older times to the most recent common ancestor \cite{rieux2016inferences}. Therefore the incorrect inclusion of incorrect sampling times, whether t<he data set is truly isochronous or not, may be result in tree extension as a means of compensating for the likelihood penalty imposed by incorrect sampling times.

While the phenomenon of tree extension occurs for the strict and relaxed molecular clock models, it is less pronounced in the relaxed molecular clock model (e.g. see Fig \ref{figure:ultrametric_tree_distortion}). A probable reason for why the relaxed molecular clock is more robust to tree extension in data sets that are truly isochronous is that this model can absorb the incorrect inclusion of sampling times by treating them as rate variation among lineages.

The presence, rather than the absence, of temporal signal is much easier to detect, meaning low type II errors. In this case, specifying an incorrect isochronous model does not result in an obvious distortion of the time tree that could mislead BETS, at least under the conditions that we used here. It is conceivable, however, that certain priors on the phylogenetic tree or evolutionary rate could result in type II errors. 

A key consideration is that the parameters of the tree prior for isochronous and heterochronous analyses should be in the same units (and of similar magnitude), which facilitates comparison of the models in question. For example, fixing the evolutionary rate to 1.0 in an isochronous analysis means that branch lengths of the time tree are in units of expected genetic distance (usually subs/site) and thus the coalescent parameters have a different meaning to those estimated under heterochronous analyses (where the evolutionary rate is a free parameter). A more tractable approach is to conduct isochronous analyses by fixing the evolutionary rate to a value in the expected order of magnitude. Biological knowledge, such as the negative correlation between genome size and rates of microbes \cite{sanjuan2012molecular, duchene2016genome} or simply a previous estimate for a closely related organism, can be helpful for specifying such values.

Our results point to a few recommendations to improve tests of temporal signal. Our first recommendation is the careful elicitation of priors, which appears important for Bayesian model selection (see \cite{bergsten2013bayesian} for a related problem in tree topology tests). We find that prior predictive simulations are essential to understand potential interactions between model parameters and ultimately whether the prior expectation is reasonable with respect to our knowledge about the data.

In our simulation study, using hard bounds on the tree height alleviated the problem of tree extension when the prior on $\theta$ favoured very old trees. In empirical data, however, one may not want to make such strong statements, and prior predictive simulations can help determine whether the tree prior and associated parameters produce trees with sensible root heights (e.g. the exponential prior on $\theta$ had low type I and type II errors in our simulations). 

Our second recommendation is to conduct prior sensitivity analyses on a set of candiate priors to determine the extent to which the prior can influence the posterior \cite{foster2017evaluating, lopes2011confronting}, and model selection using Bayes factors \cite{lambert2018student}. For example, visualising the posterior and prior distributions for a range of priors can be illuminating (see Supporting information). A prior that makes unreasonable statements and is overly influential on the posterior may need to be revised. The exponential prior on $\theta$ in the \textit{Powassan virus} did not support temporal signal, but note that it is much more influential on the posterior than the $\Gamma$ and log-normal priors for this particular data set. In contrast, the \textit{V. cholerae} data seemed robust to the three priors that we used. Some empirical data sets yielded unclear evidence for temporal signal according to BETS, such as the \textit{T. pallidium} data set that we reanalysed. In such cases, the decision of whether the data have sufficient temporal signal may require multiple lines of evidence, such as date-randomisation tests \cite{duchene2015performance, ramsden2009hantavirus}, comparisons of the prior and posterior \cite{duchene2020estimating}, and root-to-tip regressions.\cite{featherstone2023clockor2, rambaut2016exploring}. When temporal signal in inconclusive, the inclusion of additional information, such as internal node calibrations, and informative molecular clock rate priors may be essential for molecular dating.

Finally, we find that the choice of the molecular clock model has a tangible impact on tests of temporal signal. The strict molecular clock model seems to incorrectly classify data sets with no temporal signal much more often than the relaxed molecular clock model. The most likely reason for this finding is that this  mitigates tree extension when the data have no temporal signal. Thus we recommend that, where possible, the results of BETS be considered under a relaxed molecular clock model.

%Although this approach seems appropriate, the fully Bayesian nature of BETS means that it is straight-forward to incorporate additional information that may be key for obtaining temporal signal. In the study that described the \textit{T. pallidium} data set, the authors conducted their Bayesian analyses with monophyletic constraints that enforced a particular position for the root node, based on independent information \cite{majander2020ancient}, and obtained sensible results and strong temporal signal, according to a date-randomisation test. Our finding that this data set had unclear temporal signal in the absence of such constraints indicates that such additional information should be considered, where possible. - SD: left out to avoid making the wrong recommendations here. 

The sum of our results and recommendations has implications for the reliability of estimates of evolutionary rates and timescales, for determining whether a population is measurably evolving, and even for assessing the phylodynamic threshold of emerging microbes. This concept refers to the point in time when a microbe has accumulated a sufficient amount of evolutionary change to enable phylogenetic estimates of its emergence time \cite{duchene2020temporal}. Concretely, genome data from recently emerging microbes may have very few substitutions, in which case prior sensitivity analyses and prior predictive simulations would easily reveal whether the data are sufficiently informative. Overall, tests of temporal signal have a key place in genomic analyses of pathogens and our results will be useful as guidelines to improve their application and interpretation.

\section*{Materials and methods}
\subsection*{Empirical data}
We selected three different datasets to evaluate temporal signal, \textit{V. cholerae} \cite{devault2014second}, \textit{Powassan virus} \cite{vogels2023phylogeographic}, and \textit{T. pallidum} \cite{majander2020ancient}. These varied in their sampling timeframe, number of informative sites, and number of sequences. The most sparse dataset was \textit{T. pallidum}, where sampling dates spanned 481 years (1534 to 2016), with 28 sequences each containing 1,500 SNP sites. Among these samples, there were only two in the 1500's and two in the 1700's, with the rest of the samples concentrated in the 1900's to 2000's. There were 319 sequences with 11,193 sites (the complete genome) for \textit{Powassan virus}, spanning 24 years of sampling times (1995 to 2019). Most of the samples were collected after 2010, with only three isolated in 1995. For \textit{V. cholera} we had 122  sequences with 1,57 SNP sites across 73 years of sampling (1937 to 2010). There are 1,392 unique site patterns in \textit{V. cholera}, 3,457 in \textit{Powassan virus}, and 840 in \textit{T. pallidum}. 

For \textit{V. cholerae} and \textit{T. pallidum} our data consisted of SNP sites. To account for such ascertainment bias, we specified the number of constant sites of each nucleotide in our subsequent analyses. We sampled the posterior distribution using Markov chain Monte Carlo as implemented in BEAST1.10. The chain length was $10^8$ steps, sampling every $10^3$ to draw a total of $10^4$ samples. The prior for the evolutionary rate of the strict clock was a CTMC-rate reference prior (i.e. a $\Gamma(\alpha=0.5, \beta=$tree length, and mean=$\alpha/\beta$) \cite{ferreira2008bayesian}. For the log-normal distribution of the relaxed clock we also used the CTMC-rate reference prior for the mean rate (known as the ucld.mean in the program) and an exponential prior with mean 0.33 for the standard deviation (the ucld.stdev in the program). For the tree prior we used a constant-size coalescent with population size, $\theta$ with three possible priors, as described in Table \ref{table:prior_distros_on_Phi}. In all cases we used the HKY+$\Gamma_4$ substitution model, with the default priors in BEAST1.10.

To calculate log marginal likelihoods we used generalised stepping-stone \cite{baele2016genealogical,fan2011choosing}. This method requires a working distribution for all parameters, including the tree prior, for which we used the matching coalescent model. We set 100 path steps between the unnormalised posterior and the working distribution, following equally spaced intervals from a $\beta(0.3, 1.0)$ distribution. For each step we ran a chain length of 2$\times 10^{6}$ steps. We considered these settings to be appropriate after repeating the log marginal likelihood calculations 10 times for two of data sets and ensuring that the values did not vary by more than 1.0 log likelihood units.

\subsection*{Simulation experiments}
For our simulations we considered well-understood conditions to isolate the impact of the tree prior on tests of temporal signal. We simulated phylogenetic trees under a constant-size coalescent model with a fixed population size of 1.0 and a resulting average root height of 2.0 units of time. The trees could be isochronous (i.e. ultrametric) or heterochronous. In the latter case we randomly assigned tip heights of of 0.0, 0.10, 0.35, or 0.50, with equal probability. This situation implies four discrete sampling periods, which resembles sequencing blitzes \cite{porter2022new}, or archaeological sampling of strata \cite{zhang2016total}. Because the coalescent model used to analyse the data conditions on sampling times we expect it to be robust to such sampling bias \cite{stadler2015well, volz2014sampling, featherstone2021infectious}. We used a JC substitution model \cite{jukes1969evolution} to generate sequence alignments of 1,000 nucleotides and an evolutionary rate of 0.05 subs/site/time, which resulted in around 250 unique site patterns. For our simulations under a relaxed molecular clock we sampled branch rates from a log-normal distribution with mean 0.05 and standard deviation of 0.25. The procedure for obtaining the simulated alignments consisted of specifying the model above in BEAST1.10 with an empty sequence alignment to sample phylogenetic trees. We then used NELSI \cite{ho2015simulating} and Phangorn \cite{schliep2011phangorn} to simulate evolutionary rates (a single value for the strict clock and the branch rates for the relaxed molecular clock model) and sequence alignments, respectively.

We analysed the simulated data under heterochronous and isochronous models. For the heterochronous analysis we used the correct sampling times (where the correct model was the heterochronous) or assigned the tip heights above randomly (0.0, 0.10, 0.35, or 0.50) where the isocrhonous was the true model. For the evolutionary rate (clock.rate the strict and ucld.mean for the relaxed molecular clock model) we set the default CTMC-rate reference prior and the six possible configurations of the prior on $\theta$ (Table \ref{table:prior_distros_on_Phi}, plus those with hard bounds on the root height). For the isochronous analyses we did not specify sampling times and we fixed evolutionary rate to its true value of 0.05 to ensure that the branch lengths, $\theta$ and other parameters are in the correct units. We calculated log marginal likelihoods with the same procedure as for the empirical data.

\section*{Supporting information}
% Include only the SI item label in the paragraph heading. Use the \nameref{label} command to cite SI items in the text.

\paragraph*{S1 Fig.}
\label{S1_Fig}
	\begin{center}
		\includegraphics[width=13cm]{sandbox_figures/cholera_density_plot.pdf}\newline
		\textbf{Densities of key statistics of cholera empirical data.} The clock rate, root height, tree length, and coefficient of rate variation are shown under three priors on $\theta$, exponential (red), gamma (blue), and lognormal (green).
	\end{center}
\paragraph*{S2 Fig.}
\label{S2_Fig}
	\begin{center}
		\includegraphics[width=13cm]{sandbox_figures/powv_density_plot.pdf}\newline
		\textbf{Densities of key statistics of \textit{Powassan virus} empirical data.} The clock rate, root height, tree length, and coefficient of rate variation are shown under three priors on $\theta$, exponential (red), gamma (blue), and lognormal (green).
	\end{center}
\paragraph*{S3 Fig.}
\label{S3_Fig}
	\begin{center}
		\includegraphics[width=13cm]{sandbox_figures/treponema_density_plot.pdf}\newline
		\textbf{Densities of key statistics of \textit{Treponema palladium} empirical data.} The clock rate, root height, tree length, and coefficient of rate variation are shown under three priors on $\theta$, exponential (red), gamma (blue), and lognormal (green).
	\end{center}
 \paragraph*{S4 Fig.}
\label{S4_Fig}
	\begin{center}
		\includegraphics[width=13cm]{sandbox_figures/polygon_plot_bound.pdf}\newline
		\textbf{Relative log marginal likelihoods of empirical data sets with bounds on root height.} The polygons represent the relative log marginal likelihoods of each microbe dataset under a different effective population size ($\theta$) prior, analysed with four different configurations. Het (heterochronous) includes sampling, while iso (isochronous) does not include any sampling times. SC is strict clock and UCLD is the uncorrelated lognormal relaxed clock. Red represents an exponential hyperprior on the effective population size, blue is a $\Gamma$ hyperprior, and green is a log-normal hyperprior..
	\end{center}

\paragraph*{S5 Fig.}
\begin{figure}[!h]
	\begin{center}
		\includegraphics[width=14cm]{sandbox_figures/coal_exponential_het_sims.pdf}\newline
		\vspace{-0.5cm}
		\caption{\textbf{Relative log marginal likelihoods of simulations with temporal signal and analysed under an exponential-size coalescent tree prior.} The polygons represent the relative log marginal likelihood under three possible priors on the effective population size ($\theta$) parameter of the constant-size coalescent tree prior. Each corner corresponds to a combination of model and sampling times, either a strict (SC) or relaxed molecular clock with an underlying log-normal distribution (UCLD), and with (heterochronous) or without (isochronous) sampling times. The correct model used to generate the data is the SC heterochronous. Each polygon is for one simulation replicate (a total of ten) and the colours denote whether we employed a hard bound on the root height of the form Uniform(0.0, 5.0), in blue, or not, in orange.} 
		\label{figure:exponential_heterochronous_polygons}
	\end{center}
\end{figure}

\paragraph*{S6 Fig.}
\begin{figure}[!h]
	\begin{center}
		\includegraphics[width=14cm]{sandbox_figures/iso_sims.pdf}\newline
		\vspace{-0.5cm}
		\caption{\textbf{Relative log marginal likelihoods of simulations with no temporal signal and analysed under an exponential-size coalescent tree prior.} The polygons here represent the same information as in Fig \ref{figure:exponential_heterochronous_polygons}. The correct model used to generate the data here is the SC isochronous (ultrametric).} 
		\label{figure:exponential_ultrametric_polygons}
	\end{center}
\end{figure}


\section*{Acknowledgments}
This work was supported by the Inception program [Investissement d’Avenir grant ANR-16-CONV-0005], the Australian National Health and Medical Research Council [2017284], and the Australian Research Council [FT220100629].

\nolinenumbers


% Either type in your references using
% \begin{thebibliography}{}
% \bibitem{}
% Text
% \end{thebibliography}
%
% or
%
% Compile your BiBTeX database using our plos2015.bst
% style file and paste the contents of your .bbl file
% here. See http://journals.plos.org/plosone/s/latex for 
% step-by-step instructions.
% 
%\begin{thebibliography}{10}
\bibliography{References}

%\bibitem{bib1}
%Conant GC, Wolfe KH.
%\newblock {{T}urning a hobby into a job: how duplicated genes find new
%  functions}.
%\newblock Nat Rev Genet. 2008 Dec;9(12):938--950.

%\bibitem{bib2}
%Ohno S.
%\newblock Evolution by gene duplication.
%\newblock London: George Alien \& Unwin Ltd. Berlin, Heidelberg and New York:
%  Springer-Verlag.; 1970.

%\bibitem{bib3}
%Magwire MM, Bayer F, Webster CL, Cao C, Jiggins FM.
%\newblock {{S}uccessive increases in the resistance of {D}rosophila to viral
%  infection through a transposon insertion followed by a {D}uplication}.
%\newblock PLoS Genet. 2011 Oct;7(10):e1002337.

%\end{thebibliography}



\end{document}

